\documentclass[12pt, letterpaper, twoside]{article}
\usepackage{nopageno,epsfig, amsmath, amssymb}
\usepackage{physics}
\usepackage{mathtools}
\usepackage{hyperref}
\usepackage{xcolor}
\hypersetup{
    colorlinks,
    linkcolor={blue},
    citecolor={blue},
    urlcolor={blue}
}

\usepackage[letterpaper,
            margin=0.8in]{geometry}

\title{Astro 507; Problem Set 3}
\author{\textbf{Tom Wagg}}

\newcommand{\question}[1]{{\noindent \it #1}}
\newcommand{\answer}[1]{
    \par\noindent\rule{\textwidth}{0.4pt}#1\vspace{0.5cm}
}
\newcommand{\todo}[1]{{\color{red}\begin{center}TODO: #1\end{center}}}

% custom function for adding units
\makeatletter
\newcommand{\unit}[1]{%
    \,\mathrm{#1}\checknextarg}
\newcommand{\checknextarg}{\@ifnextchar\bgroup{\gobblenextarg}{}}
\newcommand{\gobblenextarg}[1]{\,\mathrm{#1}\@ifnextchar\bgroup{\gobblenextarg}{}}
\makeatother

\newcommand{\avg}[1]{\left\langle #1 \right\rangle}
\newcommand{\angstrom}{\mbox{\normalfont\AA}}
\allowdisplaybreaks

\begin{document}

\maketitle

\question{1. \textbf{FIRAS}}

\answer{}

\question{2a. \textbf{Entropy}}
\answer{
    Let's derive the entropy for an ideal, non-relativistic Fermi gas (in terms of $V$, $z$ and $T$). For this I will follow Lecture 10. The entropy is defined as
    \begin{equation}
        S = \frac{U + PV - N \mu}{T},
    \end{equation}
    so we just need to use definitions of $U$, $P$, $N$ and $\mu$. The definitions of $\mu$ is trivial when using the fugacity
    \begin{equation}
        \mu = k_B T \ln z
    \end{equation}
    For the others, I refer to lecture 10 where we can find the definition of $n$ on slide 5
    \begin{equation}
        N = n V = \frac{2(2s + 1)}{\pi^{1/2} \lambda^3} V F_{1/2}(z)
    \end{equation}
    and the definition of $P$ on slide 7
    \begin{equation}
        P = \frac{4(2s + 1)}{3 \pi^{1/2}} k_B T \lambda^{-3} F_{3/2}(z)
    \end{equation}
    For the total energy, we don't have it from the slides but we can derive it in the same way using Fermi-Dirac integrals. It starts in the same way as $n$ but with an extra factor of $\epsilon$ in the numerator, then we use the substitution $w = \epsilon / k_B T$
    \begin{align}
        U &= (2s + 1) \frac{V}{h^3} \int \frac{\epsilon}{e^{\frac{\epsilon - \mu}{k_B T}} + 1} \dd{\va{p}} \\
        U &= \frac{4\pi (2s + 1)}{2 h^3} (2 m k_B T)^{3/2} k_B T \int_0^\infty \dd{w} \frac{w^{3/2}}{e^w z^{-1} + 1} \\
        U &= \frac{2(2s + 1)}{\pi^{1/2} \lambda^3} V k_B T F_{3/2}(z)
    \end{align}
    Now it's simply a matter of plugging all of these into the entropy expression.
    \begin{align}
        S &= \frac{U + PV - N \mu}{T} \\
          &= \frac{1}{T} \qty[ \frac{2(2s + 1)}{\pi^{1/2} \lambda^3} V k_B T F_{3/2}(z) + \frac{4(2s + 1)}{3 \pi^{1/2}} V k_B T \lambda^{-3} F_{3/2}(z) + \frac{2(2s + 1)}{\pi^{1/2} \lambda^3} V k_B T \ln z F_{1/2}(z)] \\
          &= \frac{2(2s + 1) V k_B T}{\pi^{1/2} \lambda^3} \qty[ F_{3/2}(z) + \frac{2}{3} F_{3/2}(z) + \ln z F_{1/2}(z)] \\
        \Aboxed{ S &= \frac{2(2s + 1) V k_B T}{\pi^{1/2} \lambda^3} \qty[ \frac{5}{3} F_{3/2}(z) + F_{1/2}(z) \ln z] }
    \end{align}
}

\question{2b. \textbf{Expanding pressure}}
\answer{
    \begin{align}
        \frac{P}{n k_B T} &= \frac{4(2s + 1)}{3 \pi^{1/2}} k_B T \lambda^{-3} F_{3/2}(z) \qty[\frac{2(2s + 1)}{\pi^{1/2} \lambda^3} k_B T F_{1/2}(z)]^{-1} \\
                          &= \frac{2 F_{3/2}(z)}{3 F_{1/2}(z)}
    \end{align}
    \todo{come back to this on Thursday}
}

\question{\textbf{3. Brown Dwarf}}

\question{3a. \textbf{Fugacity}}
\answer{
    The general goal here is to find the number density based on the central temperature and density and then solve for the Fermi-Dirac integral and we can then invert that to find the fugacity. To start, we know that
    \begin{equation}
        \rho = m_e n_e + m_{\rm H} n_{\rm H} + m_{\rm He} n_{\rm He}
    \end{equation}
    We can rewrite this (and neglect the electron term) as
    \begin{equation}
        \rho = n_{\rm H} \qty(m_{\rm H} + m_{\rm He} \frac{n_{\rm He}}{n_{\rm H}})
    \end{equation}
    Now we can plug in the masses of hydrogen and helium as well as their relative abundance (given as 0.1 in the problem description).
    \begin{equation}
        n_{\rm H} = \frac{\rho_c}{1.4 m_p}
    \end{equation}
    Now we just relate this to the electron number density simply as
    \begin{equation}
        n = \qty(2 \frac{n_{\rm He}}{n_{\rm H}} + 1) n_{\rm H}
    \end{equation}
    And combining the two gives an expression for the number density in terms of the central density
    \begin{equation}
        n = \frac{1.2 \rho_c}{1.4 m_p}
    \end{equation}
    The other part that we need is the de-Broglie wavelength which depends on the central temperature as
    \begin{equation}
        \lambda = \frac{h}{\sqrt{2 \pi m k_B T_c}}
    \end{equation}
    Now let's use this with the equation for the number density from Lecture 10 and solve for the Fermi-Dirac integral (plugging in values in the final equation).
    \begin{align}
        n &= \frac{2(2s + 1)}{n \pi^{1/2} \lambda^3} F_{1/2}(z) \\
        F_{1/2}(z) &= \frac{n \pi^{1/2} \lambda^3}{2(2s + 1)} \\
        F_{1/2}(z) &= \frac{\pi^{1/2}}{2(2s + 1)} \frac{1.2 \rho_c}{1.4 m_p} \qty(\frac{h}{\sqrt{2 \pi m k_B T_c}})^3 \\
        F_{1/2}(z) &= 2.08
    \end{align}
    I took this value and inverted it using the approximate analytic expressions from the paper (Aymerich-Humet+1981) to find the fugacity.
    \begin{equation}
        \boxed{ z = 1.65 }
    \end{equation}
}

\end{document}

 